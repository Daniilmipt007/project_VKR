\documentclass[a4paper, 12pt]{extarticle}


\usepackage[T2A]{fontenc}
\usepackage[utf8]{inputenc}
\usepackage[english,russian]{babel}
% \usepackage{cmap}
\usepackage{url}
\usepackage{booktabs}
\usepackage{nicefrac}
\usepackage{microtype}
\usepackage{lipsum}
\usepackage[square,sort,comma,numbers]{natbib}
\usepackage{doi}
\usepackage{multicol}
\usepackage{multirow}
\usepackage{tabularx}
\usepackage{amsmath,bm}
\usepackage{tikz}
\usetikzlibrary{matrix}

% Algorithms
\usepackage{algpseudocode}
\usepackage{algorithm}

%% Шрифты
\usepackage{euscript} % Шрифт Евклид
\usepackage{mathrsfs} % Красивый матшрифт
\usepackage{extsizes}

\usepackage{makecell} % diaghead in a table
\usepackage{amsmath,amsfonts,amssymb,amsthm,mathtools,dsfont}
\usepackage{icomma}

\usepackage{hyperref}
% \usepackage[usenames,dvipsnames,svgnames,table,rgb]{xcolor}

\hypersetup{
	unicode=true,
	pdftitle={пространственно-временные характеристики в задаче декодирования временных рядов},
	pdfauthor={Дорин Даниил Дмитриевич},
	pdfkeywords={},
	colorlinks=true,
	linkcolor=black,        % внутренние ссылки
	citecolor=blue,         % на библиографию
	urlcolor=magenta           % на URL
}

\definecolor{linkcolor}{rgb}{0.9,0,0}
\definecolor{citecolor}{rgb}{0,0.6,0}
\definecolor{urlcolor}{rgb}{0,0,1}

\graphicspath{{figs}}

\usepackage{enumitem} % Для модификаций перечневых окружений
\usepackage{etoolbox}
\usepackage{amsmath,amssymb}
\usepackage{accents}
\makeatletter
\expandafter\patchcmd\csname\string\algorithmic\endcsname{\itemsep\z@}{\itemsep=1.5mm}{}{}
\makeatother
\newcommand{\myfigref}[2]{~\ref{#1}.\subref{#2}}% <---- a new macro for referring to a subfigure
\usepackage{geometry}
\geometry{a4paper, top=2cm, bottom=2cm, left=2.5cm, right=1cm}
\setlength\parindent{5ex}    % Устанавливает длину красной строки 15pt
%\linespread{1.3}             % Коэффициент межстрочного интервала
\usepackage{setspace}
%\usepackage[center]{titlesec} % секции посередине
\usepackage{subfig}
\usepackage{graphicx}
%\usepackage{subcaption}
\captionsetup[subfigure]{font={small},labelfont={small}}
%%% Точки после номеров разделов
\renewcommand{\thesection}{\arabic{section}.}
\renewcommand{\thesubsection}{\arabic{section}.\arabic{subsection}.}
\renewcommand{\thesubsubsection}{\arabic{section}.\arabic{subsection}.\arabic{subsubsection}.}
%%% Лишнее расстояние
%\titlelabel{\thetitle \ }
\usepackage{amsmath}               
  {
      \theoremstyle{plain}
      \newtheorem{assumption}{Предположение}
  }
\usepackage{icomma}
\usepackage{amsthm}
\usepackage{amssymb}
\usepackage{amsmath}
\usepackage{color}
%\usepackage{bm}
\usepackage{tabularx}
\usepackage{url}
\usepackage{multirow}
\usepackage{wrapfig}
\usepackage{caption}
\usepackage{indentfirst}

% Теоремы
\newtheorem{theorem}{Теорема}
\newtheorem{lemma}{Лемма}
\newtheorem{proposition}{Утверждение}
\newtheorem*{exercise}{Упражнение}
\newtheorem*{problem}{Задача}

\newtheorem{definition}{Определение}
\newtheorem*{corollary}{Следствие}
\newtheorem*{note}{Замечание}
\newtheorem*{reminder}{Напоминание}
\newtheorem*{example}{Пример}
\newtheorem*{cexample}{Контрпример}
\newtheorem*{solution}{Решение}

\renewcommand{\abstractname}{Аннотация}
\renewcommand{\abstractname}{Аннотация}

\title{Пространственно-временные характеристики в задаче декодирования временных рядов.}

\author{
	Дорин Даниил \\
	\texttt{dorin.dd@phystech.edu} \\
	\And
	Грабовой Андрей \\
	\texttt{grabovoy.av@phystech.edu}
}
\date{\today}

\begin{document}
\maketitle

\begin{abstract}

	Исследуется проблема нахождения пространственно-временных характеристик в задаче декодирования временных рядов. В задачах декодирования сигнала данные представляются как многомерные временные ряды с дискретным представлением времени. 
	Проводится обзор методов анализа пространственно-временных характеристик нескольких временных рядов.
    Предложен ...
	Для анализа предложенного метода проводится вычислительный эксперимент на
	выборке \citep{misc_eeg_eye_state_264}, полученной при исследования электрической активности мозга большого числа испытуемых с помощью инвазивной электроэнцефалографии.

\end{abstract}


\keywords{ЭЭГ \and временные ряды \and }

\section{Введение}
Человеческий мозг~--- один из самых интересных объектов исследования \citep{Zhumakova}. 
Внутречерепные записи человека являются редким и ценным источником информации о мозге.

\section{Постановка задачи}
Исследуется задача декодирования временного ряда. Пусть имеется некоторый процесс (активность головного мозга):
$$\mathcal{V}(\tau),~\tau \in \mathbb{R}$$
Тогда данные выборки ~--- это регестрируемый сигнал, то есть реализация процесса $\mathcal{V}(\tau)$:
$$\bm{X} = \left[\bm{x}_1,\dots \bm{x}_{T}\right]^{\T},~\bm{x}_t \in \mathbb{R}^K$$
Здесь $K$ ~--- число каналов. $T$ ~--- число измерений сигнала с частотой $\mu$ за время $\tau$:
$$T = \tau \mu$$
$$\bm{x}_{\tau \mu} \approx \mathcal{V}(\tau)$$
\subsection{Задача классификации сигнала}
Пусть имеется выборка из $N$ наблюдений:
$$\bm{X} = \{\bm{X_1},\dots, \bm{X_N}\},$$
$$\bm{X}_i = \left[\bm{x}^i_1,\dots \bm{x}^i_{T_i}\right]^{\T}, ~\bm{x}^i_t \in \mathbb{R}^K,$$
$$\bm{Y} = \{\bm{Y}_1,\dots, \bm{Y}_N\},~\bm{Y}_i = \left[y_1, \dots, y_{T_i}\right]^{\T},~y_t \in \{1,\dots, C\}$$
Здесь $C$ ~--- число классов в задаче классификации сигнала. 

Для набора данных, описанного выше, требуется построить отображение $f_\theta$, которое учитывало 
бы пространственно-временные характеристиик между временными рядами сигнала:
$$f_\theta: \bm{X} \rightarrow \bm{Y}$$ 



\section{Вычислительный эксперимент}

Для анализа работоспособности предложенного метода, а также проверки гипотез
проведен вычислительный эксперимент.

Для проведения экспериментов была использована выборка бинарной классификации состояния глаз 
испытуемого (открыты или закрыты), представленная в \citep{misc_eeg_eye_state_264}.

Набор данных получен в результате одного непрерывного измерения неинвазивного ЭЭГ с 
помощью нейроголовки Emotiv EEG с использованием 14 датчиков, на рис.\ref{fig:1} задействованные датчики
изображены красным цветом. 
\begin{figure}[h]
	\centering
	\includegraphics[width=0.75\textwidth]{64_channel_sharbrough.pdf}
	\caption{Задействованные датчики ЭЭГ при измерении сигнала}
	\label{fig:1}
\end{figure}


Продолжительность измерения в выборке составила 117 секунд. 
Состояние глаз было зафиксировано с помощью камеры во время измерения ЭЭГ и позже добавлено 
вручную в файл после анализа видеокадров. Метка <<1>> указывает на состояние с закрытыми глазами, а 
<<0>> ~--- на состояние с открытыми глазами. Все значения приведены в хронологическом порядке с 
первым измеренным значением в верхней части данных.
Основные характеристики выборки представлены в
Таблице~\ref{table:sample}.

\begin{table}
	\centering
	\caption{Описание выборки}
	\begin{tabular}{|c|c|c|}
		\hline
		Название                       & Обозначение & Значение             \\
		\hline \hline
		Продолжительность обследования & $\tau$         & 117 с                \\ \hline
		Частота измерения сигнала      & $\mu$       & 128.03 $\text{с}^{-1}$   \\ \hline
	    Число каналов (датчиков)    & $K$   & 14          \\ \hline
		Число измерений сигнала             & $T$  & 14980           \\ \hline
	\end{tabular}
	\label{table:sample}
\end{table}
При анализе выборки обнаружено 4 выброса, которые были заменены средними по классам 
значениями, график временных рядов после обработки выбросов представлен на рис.\ref{fig:2}.

\begin{figure}[h]
	\centering
	\includegraphics[width=0.8\textwidth]{Dataset.png}
	\caption{График временных рядов}
	\label{fig:2}
\end{figure}



\section{Заключение}


\newpage

\bibliographystyle{plain}
\bibliography{references.bib}

\end{document}
