\documentclass[a4paper, 12pt]{extarticle}


\usepackage[T2A]{fontenc}
\usepackage[utf8]{inputenc}
\usepackage[english,russian]{babel}
% \usepackage{cmap}
\usepackage{url}
\usepackage{booktabs}
\usepackage{nicefrac}
\usepackage{microtype}
\usepackage{lipsum}
\usepackage[square,sort,comma,numbers]{natbib}
\usepackage{doi}
\usepackage{multicol}
\usepackage{multirow}
\usepackage{tabularx}
\usepackage{amsmath,bm}
\usepackage{tikz}
\usetikzlibrary{matrix}

% Algorithms
\usepackage{algpseudocode}
\usepackage{algorithm}

%% Шрифты
\usepackage{euscript} % Шрифт Евклид
\usepackage{mathrsfs} % Красивый матшрифт
\usepackage{extsizes}

\usepackage{makecell} % diaghead in a table
\usepackage{amsmath,amsfonts,amssymb,amsthm,mathtools,dsfont}
\usepackage{icomma}

\usepackage{hyperref}
% \usepackage[usenames,dvipsnames,svgnames,table,rgb]{xcolor}

\hypersetup{
	unicode=true,
	pdftitle={пространственно-временные характеристики в задаче декодирования временных рядов},
	pdfauthor={Дорин Даниил Дмитриевич},
	pdfkeywords={},
	colorlinks=true,
	linkcolor=black,        % внутренние ссылки
	citecolor=blue,         % на библиографию
	urlcolor=magenta           % на URL
}

\definecolor{linkcolor}{rgb}{0.9,0,0}
\definecolor{citecolor}{rgb}{0,0.6,0}
\definecolor{urlcolor}{rgb}{0,0,1}

\graphicspath{{figs}}

\usepackage{enumitem} % Для модификаций перечневых окружений
\usepackage{etoolbox}
\usepackage{amsmath,amssymb}
\usepackage{accents}
\makeatletter
\expandafter\patchcmd\csname\string\algorithmic\endcsname{\itemsep\z@}{\itemsep=1.5mm}{}{}
\makeatother
\newcommand{\myfigref}[2]{~\ref{#1}.\subref{#2}}% <---- a new macro for referring to a subfigure
\usepackage{geometry}
\geometry{a4paper, top=2cm, bottom=2cm, left=2.5cm, right=1cm}
\setlength\parindent{5ex}    % Устанавливает длину красной строки 15pt
%\linespread{1.3}             % Коэффициент межстрочного интервала
\usepackage{setspace}
%\usepackage[center]{titlesec} % секции посередине
\usepackage{subfig}
\usepackage{graphicx}
%\usepackage{subcaption}
\captionsetup[subfigure]{font={small},labelfont={small}}
%%% Точки после номеров разделов
\renewcommand{\thesection}{\arabic{section}.}
\renewcommand{\thesubsection}{\arabic{section}.\arabic{subsection}.}
\renewcommand{\thesubsubsection}{\arabic{section}.\arabic{subsection}.\arabic{subsubsection}.}
%%% Лишнее расстояние
%\titlelabel{\thetitle \ }
\usepackage{amsmath}               
  {
      \theoremstyle{plain}
      \newtheorem{assumption}{Предположение}
  }
\usepackage{icomma}
\usepackage{amsthm}
\usepackage{amssymb}
\usepackage{amsmath}
\usepackage{color}
%\usepackage{bm}
\usepackage{tabularx}
\usepackage{url}
\usepackage{multirow}
\usepackage{wrapfig}
\usepackage{caption}
\usepackage{indentfirst}

% Теоремы
\newtheorem{theorem}{Теорема}
\newtheorem{lemma}{Лемма}
\newtheorem{proposition}{Утверждение}
\newtheorem*{exercise}{Упражнение}
\newtheorem*{problem}{Задача}

\newtheorem{definition}{Определение}
\newtheorem*{corollary}{Следствие}
\newtheorem*{note}{Замечание}
\newtheorem*{reminder}{Напоминание}
\newtheorem*{example}{Пример}
\newtheorem*{cexample}{Контрпример}
\newtheorem*{solution}{Решение}

\renewcommand{\abstractname}{Аннотация}
\renewcommand{\abstractname}{Аннотация}

\title{Пространственно-временные характеристики в задаче декодирования временных рядов.}

\author{
	Дорин Даниил \\
	\texttt{dorin.dd@phystech.edu} \\
	\And
	Грабовой Андрей \\
	\texttt{grabovoy.av@phystech.edu}
}
\date{\today}

\begin{document}
\maketitle

\begin{abstract}

	Исследуется пространственно-временные характеристики в задаче декодирования временных рядов. В проблеме декодирования 
	сигнала данные представляются как многомерные временные ряды с дискретным представлением времени. 
	В работе приведен обзор методов классификации сигнала в задаче декодирования временных рядов.
    Предложен метод классификации временных рядов, основанный на Римановой геометрии.
	Для анализа предложенного метода проводится вычислительный эксперимент на
	выборке \citep{misc_eeg_eye_state_264}, полученной при исследования электрической активности мозга большого с помощью неинвазивной электроэнцефалографии.

\end{abstract}

\keywords{ЭЭГ \and временные ряды \and Риманова геометрия}

\section{Введение}
Основной целью анализа сигнала в данном исследовании является 
классификация электроэнцефалограммы (ЭЭГ)\citep{teplan2002fundamentals, beniczky2020electroencephalography}~--- раздел электрофизиологии, 
изучающий закономерности суммарной электрической активности мозга, 
отводимой с поверхности кожи волосистой части головы, 
а также метод записи таких потенциалов. Также ЭЭГ ~--- неинвазивный метод, то есть 
не требует проникновения внутрь организма или повреждения кожи или других тканей. 
Вместо этого, данные собираются с помощью внешних средств. 
В последнее время активно ведутся научные исследования, 
посвященные методам регисттрации активности мозга и декодированию 
информации \citep{siuly2016eeg, craik2019deep}. Основным направлением применения 
этих методов являются технологии нейрокомпьютерных интерфейсов.

Одним из успешных традиционных методов классификации наличия потенциала P300 на электроэнцефалограммы (ЭЭГ) является
алгоритм \textbf{ERPCov TS LR}. Под потенциалом P300 понимается связанный с 
событием (event-related potential — ERP) измеренный отклик мозга, 
который является прямым результатом определенного ощущения, 
когнитивного или моторного события.
Алгоритм основан на применении Римановой геометрии\citep{barachant2010riemannian}. 
Первым этапом данного алгоритма является формирование пространства центрированных признаков.

\begin{equation*}
	\bm{X}_i= \left[\bm{x}^i_1,\dots \bm{x}^i_{T}\right] = 
	\begin{bmatrix}
	x^i_{1,1} & x^i_{1,2} &  \dots  &  x^i_{1,T}\\
	\dots & \dots &  \dots  &  \dots\\
	x^i_{K,1} & x^i_{K,2} &  \dots  &  x^i_{K,T}
	\end{bmatrix}
	= \begin{bmatrix}
		ts_1\\
		\dots\\
		ts_K
		\end{bmatrix},
	\end{equation*}
где $ts_j$ ~--- временной ряд с нулевым средним, полученный при измерении сигнала $j$-ым датчиком и последующего центрирования.
Тогда ковариационная матрица для одного измерения ЭЭГ имеет вид:
$$\bm{R}_i = \dfrac{1}{T-1}\bm{X}_i\bm{X}_i^{\T}, ~\bm{R} \in \mathbb{R}^{K\times K},~i = \overline{1, N}$$

Для классификации потенциала P300 в алгоритме используется расширенная матрица ковариации:
$$\bm{R}_i = \dfrac{1}{T-1} \bm{P}_i\bm{P}_i^{\T},~\bm{P}_i = \begin{bmatrix}
	\overline{\bm{X}^0}\\
	\overline{\bm{X}^1}\\
	\bm{X}_i
	\end{bmatrix},$$
где $\overline{\bm{X}^c}$ и $\overline{\bm{X}^1}$~--- средние по классам $\{0,1\}$ значения:
$$\overline{\bm{X}^0} = \dfrac{\sum_{i = 1}^N\left[y_i = c\right] \bm{X}_i}{\sum_{i = 1}^N\left[y_i = c\right]},~c\in\{0,1\}$$
Известно, что пространство, состоящее из матриц ковариации, представляет собой 
риманово многообразие\citep{barachant2010riemannian}. 
В каждой точке данного риманова многообразия имеется касательная плоскость с 
определенным скалярным произведением на ней. Общая касательная плоскость, 
предназначенная для отображения всех матриц ковариации в выборке, 
формируется в точке среднего геометрического по римановой метрике 
известных ковариационных матриц. Среднее геометрическое симметричных положительно определенных матриц имеет вид:
$$\bm{R} = \mathfrak{G}\left(\bm{R}_1,\dots,\bm{R}_N\right) = \underset{\bm{R}}{\text{argmin}}\sum_{i = 1}^N
\delta^2_R(\bm{R},~\bm{R}_i),$$
где риманова метрика определяется следующим образом (геодезическое расстояние):
$$\delta_R(\bm{R},~\bm{R}_i) = \|\log (\bm{R}^{-1}\bm{R}_i)\|_F = \sqrt{\sum_{i = 1}^{3N} \log^2\lambda_i},$$
где $\lambda_i$~--- собственные значения матрицы $\bm{R}^{-1}\bm{R}_i$. В работе \citep{barachant2010riemannian}
получено, что для каждой ковариационной матрицы $\bm{R}_i$ существует проекция $\bm{\pi}_i$ на касательное пространство.
Таким образом, определено отображение:
$$\text{Exp}_{R}(\bm{\pi}_i) = \bm{R}_i = \bm{R}^{\frac{1}{2}} \exp\left(\bm{R}^{-\frac{1}{2}}\bm{\pi}_i\bm{R}^{-\frac{1}{2}}\right) \bm{R}^{\frac{1}{2}}$$
$$\log_{R}(\bm{R}_i) = \bm{\pi}_i = \bm{R}^{\frac{1}{2}} \log\left(\bm{R}^{-\frac{1}{2}}\bm{R}_i\bm{R}^{-\frac{1}{2}}\right) \bm{R}^{\frac{1}{2}}$$

На практике построение ковариационных матриц и получение их образов в касательном пространстве выполняется 
при помощи библиотеки \textbf{PyRiemann} \citep{congedo2013new}.

После векторизации полученные данные используются как новое признаковое пространство и 
могут быть классифицированы при помощи алгоритма логистической регрессии. 

%%%%%%%%%%%%%%%%%%%%%%%%%%%%%%%%%%%%%%%%%%%%%%%%%%%%%%%%%%%%%%%%%%%%%%%%%%%%%%%%%%%%%%%%%
\section{Постановка задачи}
Исследуется задача декодирования временного ряда. Пусть имеется некоторый процесс (активность головного мозга):
$$\mathcal{V}(\tau),~\tau \in \mathbb{R}$$
Тогда данные выборки ~--- это регистрируемый сигнал, то есть реализация процесса $\mathcal{V}(\tau)$:
$$\bm{X} = \left[\bm{x}_1,\dots \bm{x}_{T}\right],~\bm{x}_t \in \mathbb{R}^K$$
Здесь $K$ ~--- число каналов. $T$ ~--- число измерений сигнала с частотой $\mu$ за время $\tau$:
$$T = \tau \mu$$
$$\bm{x}_{\tau \mu} \approx \mathcal{V}(\tau)$$
\subsection{Задача классификации отрезков регистрируемого сигнала}
В данной задаче имеется выборка регистрируемых отрезков сигнала, 
требуется классифицировать каждый наблюдаемый временной отрезок. 
Введем следующие обозначения:
Пусть имеется $N$ зарегистрированных реализаций некоторого процесса:
$$\bm{X} = \{\bm{X}_1,\dots, \bm{X}_N\},$$
$$\bm{X}_i = \left[\bm{x}^i_1,\dots, \bm{x}^i_{T}\right], ~\bm{x}^i_t \in \mathbb{R}^K,$$
$$\bm{Y} = \left[y_1, \dots, y_{N}\right]^{\T},~y_i \in \{1,\dots, C\}$$
Здесь $y_i$~--- целевая метка класса $i$-го зарегистрированного сигнала. $C$ ~--- число классов в задаче классификации сигнала. 

Имеется соответственно выборка $\mathcal{D} = \{y_i, \bm{X}_i\},~  i = \overline{1,N}$
Требуется построить отображение $f_\theta$, которое учитывало 
бы пространственно-временные характеристиик между временными рядами от датчиков:
$$f_\theta: \bm{X} \rightarrow \{1,\dots, C\}$$ 
\subsection{Задача классификации активности}
В данной задаче предполагается получение классификации для каждого отсчета 
времени наблюдения.
Пусть имеется некоторый процесс и зарегистрированная реализация данного 
процесса в виде дискретного числа измерений. Каждому измерению соответствует
класс активности. Формально:
$$\bm{X} = \{\bm{x}_1,\dots, \bm{x}_{T}\}, ~\bm{x}_t \in \mathbb{R}^K,$$
$$\bm{Y} = \left[y_1, \dots, y_{T}\right]^{\T},~y_t \in \{1,\dots, C\}$$
Здесь $C$ ~--- число классов в задаче классификации активности. 
Выборка $\mathcal{D} = \{y_t, \bm{x}_t\}_{t=1}^T$

Для набора данных, описанного выше, требуется построить отображение $f_\theta$, которое учитывало 
бы пространственно-временные характеристиик между временными рядами сигнала:
$$f_\theta: \bm{X} \rightarrow \{1,\dots, C\}$$ 




\section{Вычислительный эксперимент}

Для анализа работоспособности предложенного метода, а также проверки гипотез
проведен вычислительный эксперимент.

Для проведения экспериментов была использована выборка бинарной классификации состояния глаз 
испытуемого (открыты или закрыты), представленная в \citep{misc_eeg_eye_state_264}.

Набор данных получен в результате одного непрерывного измерения неинвазивного ЭЭГ с 
помощью нейроголовки Emotiv EEG с использованием 14 датчиков, на рис.\ref{fig:1} задействованные датчики
изображены красным цветом. 
\begin{figure}[h]
	\centering
	\includegraphics[width=0.75\textwidth]{64_channel_sharbrough.pdf}
	\caption{Задействованные датчики ЭЭГ при измерении сигнала}
	\label{fig:1}
\end{figure}


Продолжительность измерения в выборке составила 117 секунд. 
Состояние глаз было зафиксировано с помощью камеры во время измерения ЭЭГ и позже добавлено 
вручную в файл после анализа видеокадров. Метка <<1>> указывает на состояние с закрытыми глазами, а 
<<0>> ~--- на состояние с открытыми глазами. Все значения приведены в хронологическом порядке с 
первым измеренным значением в верхней части данных.
Основные характеристики выборки представлены в
Таблице~\ref{table:sample}.

\begin{table}
	\centering
	\caption{Описание выборки}
	\begin{tabular}{|c|c|c|}
		\hline
		Название                       & Обозначение & Значение             \\
		\hline \hline
		Продолжительность обследования & $\tau$         & 117 с                \\ \hline
		Частота измерения сигнала      & $\mu$       & 128.03 $\text{с}^{-1}$   \\ \hline
	    Число каналов (датчиков)    & $K$   & 14          \\ \hline
		Число измерений сигнала             & $T$  & 14980           \\ \hline
	\end{tabular}
	\label{table:sample}
\end{table}
При анализе выборки обнаружено 4 выброса, которые были заменены средними по классам 
значениями, график временных рядов после обработки выбросов представлен на рис.\ref{fig:2}.

\begin{figure}[h]
	\centering
	\includegraphics[width=0.8\textwidth]{Dataset.pdf}
	\caption{График временных рядов}
	\label{fig:2}
\end{figure}



\section{Заключение}


\newpage

\bibliographystyle{plain}
\bibliography{references.bib}

\end{document}
