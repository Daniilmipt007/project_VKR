\usepackage[T2A]{fontenc}
\usepackage[utf8]{inputenc}
\usepackage[english,russian]{babel}
% \usepackage{cmap}
\usepackage{url}
\usepackage{booktabs}
\usepackage{nicefrac}
\usepackage{microtype}
\usepackage{lipsum}
\usepackage[square,sort,comma,numbers]{natbib}
\usepackage{doi}
\usepackage{multicol}
\usepackage{multirow}
\usepackage{tabularx}
\usepackage{amsmath,bm}
\usepackage{tikz}
\usetikzlibrary{matrix}

% Algorithms
\usepackage{algpseudocode}
\usepackage{algorithm}

%% Шрифты
\usepackage{euscript} % Шрифт Евклид
\usepackage{mathrsfs} % Красивый матшрифт
\usepackage{extsizes}

\usepackage{makecell} % diaghead in a table
\usepackage{amsmath,amsfonts,amssymb,amsthm,mathtools,dsfont}
\usepackage{icomma}

\usepackage{hyperref}
% \usepackage[usenames,dvipsnames,svgnames,table,rgb]{xcolor}

\hypersetup{
	unicode=true,
	pdftitle={пространственно-временные характеристики в задаче декодирования временных рядов},
	pdfauthor={Дорин Даниил Дмитриевич},
	pdfkeywords={},
	colorlinks=true,
	linkcolor=black,        % внутренние ссылки
	citecolor=blue,         % на библиографию
	urlcolor=magenta           % на URL
}

\definecolor{linkcolor}{rgb}{0.9,0,0}
\definecolor{citecolor}{rgb}{0,0.6,0}
\definecolor{urlcolor}{rgb}{0,0,1}

\graphicspath{{figs}}

\usepackage{enumitem} % Для модификаций перечневых окружений
\usepackage{etoolbox}
\usepackage{amsmath,amssymb}
\usepackage{accents}
\makeatletter
\expandafter\patchcmd\csname\string\algorithmic\endcsname{\itemsep\z@}{\itemsep=1.5mm}{}{}
\makeatother
\newcommand{\myfigref}[2]{~\ref{#1}.\subref{#2}}% <---- a new macro for referring to a subfigure
\usepackage{geometry}
\geometry{a4paper, top=2cm, bottom=2cm, left=2.5cm, right=1cm}
\setlength\parindent{5ex}    % Устанавливает длину красной строки 15pt
%\linespread{1.3}             % Коэффициент межстрочного интервала
\usepackage{setspace}
%\usepackage[center]{titlesec} % секции посередине
\usepackage{subfig}
\usepackage{graphicx}
%\usepackage{subcaption}
\captionsetup[subfigure]{font={small},labelfont={small}}
%%% Точки после номеров разделов
\renewcommand{\thesection}{\arabic{section}.}
\renewcommand{\thesubsection}{\arabic{section}.\arabic{subsection}.}
\renewcommand{\thesubsubsection}{\arabic{section}.\arabic{subsection}.\arabic{subsubsection}.}
%%% Лишнее расстояние
%\titlelabel{\thetitle \ }
\usepackage{amsmath}               
  {
      \theoremstyle{plain}
      \newtheorem{assumption}{Предположение}
  }
\usepackage{icomma}
\usepackage{amsthm}
\usepackage{amssymb}
\usepackage{amsmath}
\usepackage{color}
%\usepackage{bm}
\usepackage{tabularx}
\usepackage{url}
\usepackage{multirow}
\usepackage{wrapfig}
\usepackage{caption}
\usepackage{indentfirst}

% Теоремы
\newtheorem{theorem}{Теорема}
\newtheorem{lemma}{Лемма}
\newtheorem{proposition}{Утверждение}
\newtheorem*{exercise}{Упражнение}
\newtheorem*{problem}{Задача}

\newtheorem{definition}{Определение}
\newtheorem*{corollary}{Следствие}
\newtheorem*{note}{Замечание}
\newtheorem*{reminder}{Напоминание}
\newtheorem*{example}{Пример}
\newtheorem*{cexample}{Контрпример}
\newtheorem*{solution}{Решение}

\renewcommand{\abstractname}{Аннотация}