\begin{center}
    \Large{\textbf{Аннотация}}
\end{center}

Исследуются пространственно-временные характеристики в задаче декодирования временных рядов с дискретным представлением времени. Проводится анализ временной зависимости между последовательностью снимков функциональной магнитно-резонансной томографии, далее фМРТ, и видеорядом, просматриваемым человеком. Предлагается метод аппроксимации показаний фМРТ по просматриваемому видеоряду.
Метод строится в предположении наличия не изменяющегося времени гемодинамической ответной реакции зависимости уровня кислорода в крови. Для каждого вокселя снимка фМРТ независимо строится линейная модель. Используется предположение марковости последовательности снимков фМРТ.
Для анализа предложенного метода проводится вычислительный эксперимент на выборке, полученной при томографическом обследовании большого числа испытуемых.
Проводится анализ областей мозга, реагирующих на визуальные стимулы.
Предлагается метод снижения пространственной размерности временных рядов фМРТ, основанный на взвешивании активных областей мозга.
Построен метод классификации сегментов временных рядов фМРТ, основанный на снижении пространственной размерности и применении римановой геометрии. Проводится вычислительный эксперимент для проверки работоспособности модели. Анализируется влияние отдельных компонент метода на качество классификации.