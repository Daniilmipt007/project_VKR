\section{Заключение}                        % Заголовок
%\addcontentsline{toc}{section}{Заключение}    % Добавляем его в оглавление
В работе исследуются пространственно-временные характеристики в задаче декодирования временных рядов.
Предлагается метод аппроксимации последовательности снимков фМРТ по видеоряду. 
Метод учитывает время гемодинамической ответной реакции~--- время задержки между изображениями из видеоряда и снимками фМРТ. 
Для каждого вокселя снимка фМРТ независимо строится линейная модель. 
Каждая линейная модель строится в предположении марковости последовательности снимков фМРТ. 
В ходе экспериментов для каждого испытуемого подбирается оптимальное значение времени задержки. 
Оптимальное значение находится из анализа графика зависимости MSE от времени задержи.
Подбирается коэффициент регуляризации. 
Исследуется влияние коэффициента сжатия снимков фМРТ на время обучения модели.
Предполагается, что за информацию со зрительных органов отвечает затылочная доля мозга.
Производится поправка MSE на основе локализации этой области и выбора наиболее изменяющихся вокселей. 
При таком построении график имеет характерный минимум, отвечающий оптимальному значению времени задержки.
Полученное значение времени задержки согласуется с нейробиологическими сведениями.
Экспериментальные значения MSE малы, что говорит о наличии корреляции между данными. 
Учитывается изменение изображений в видеоряде, так как распределение весов модели не вырождено.
Проверена гипотеза инвариантности весов модели относительно человека. 
Корректность метода подтверждается экспериментами со случайными данными. 

Кроме того, предложен метод взвешивания активных областей мозга в задаче декодирования временных рядов фМРТ. Демонстрируется качество работы метода на реальных данных фМРТ. Полученные методом области близки к разметке нейробиологов. Проверяется статистическая значимость взвешенных вокселей. Корректность метода подтверждается экспериментом со случайным рядом стимула. В работе предложен метод классификации сегментов временного ряда фМРТ. Данный метод учитывает пространственно-временные характеристики благодаря применению масок активности головного мозга и извлечению признаков с помощью метода \textit{Tangent Space Mapping}, основанного на римановой геометрии. При проведении экспериментов было выявлено, что исключение отдельных компонент метода приводит к значительному снижению качества классификации. Это свидетельствует о том, что пространственно-временные характеристики, извлекаемые данным методом, являются важными для достижения высокой точности классификации.
