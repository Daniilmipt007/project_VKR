\section{Введение}\label{intro}                 % Заголовок
Совокупность методов, визуализирующих структуру и функции человеческого мозга,
называется \textit{нейровизуализацией}. Методы нейровизуализации \cite{puras2014neurovisualization} такие, как ЭКГ, КТ, МРТ и фМРТ,
используются для изучения мозга, а также для обнаружения заболеваний и психических расстройств. В рамках данной работы, основным объектом исследования являются временные ряды фМРТ.

\textit{Функциональная магнитно-резонансная томография} или \textit{фМРТ}
является разновидностью магнитно-резонансной томографии и основана на изменениях в токе крови,
вызванных нейронной активностью мозга \cite{Glover2011}.
Эти изменения происходят не моментально, а с некоторой задержкой,
которая составляет 4--8 с \cite{Bandettini1992}.
Она возникает из-за того, что сосудистая система достаточно долго реагирует
на потребность мозга в глюкозе \cite{Ogawa1990, LEBIHAN1995231, Logothetis2003}.

При получении снимков фМРТ используются последовательности
эхопланарных изображений (EPI) \cite{Connelly1993, Kwong1992, Ogawa1992}.
Обработка участков с изменяющейся интенсивностью сигнала в
зависимости от способа активации, вида артефактов и длительности
проводится с помощью специальных методов и программ
\cite{Bandettini1992, BAUDENDISTEL1995701, COX1996162}.
Обработанные результаты оформляются в виде карт активации,
которые совмещаются с локализацией анатомических образований
головного мозга.

Метод фМРТ играет большую роль в нейровизуализации, однако имеет ряд важных ограничений.
В работах \cite{menon1999spatial, logothetis2008we} рассматриваются
временное и пространственное разрешения фМРТ. Временное разрешение является существенным
недостатком данного метода. Другой недостаток фМРТ~--- неизбежно возникающие шумы,
связанные с движением объекта в сканере, сердцебиением и дыханием человека, тепловыми
флуктуациями самого прибора и т.\,д. В работе \cite{1804.10167} предлагаются методы
подавления вышеперечисленных шумов на основе графов и демонстрируется их эффективность в задаче
выявления эпилепсии и депрессии.

При проведении фМРТ испытуемому дают различные тест-задания и
применяют внешние раздражители, вызывающие активацию определенных
локальных участков головного мозга, ответственных за выполнение
соответствующих функций.
Среди тест-заданий зачастую выделяют движения пальцами и конечностями
\cite{Roux1998, Papke1999}, просмотр изображений различных категорий и рассмотрение
шахматной доски \cite{Engel1994, Schneider1994},
прослушивание неспецифичных шумов, единичных слов
или связного текста \cite{Binder1994, Dymarkowski1998}.
Причиной изменения активности человеческого мозга во время фМРТ-обследования
также может служить просмотр видеоматериала \cite{decety1997brain},
что является одним из объектов исследования настоящей работы.

Наиболее известные методы обработки видео основаны на 3D свертках \cite{tran2015learning}.
Отличие 3D от 2D сверток заключается в одновременной работе с пространственной и временной частью
информации. Существенный недостаток данных методов — сильное увеличение числа параметров модели и
большие вычислительные затраты. Одной из наиболее современных и улучшаемых архитектур
нейронных сетей для обработки изображений является остаточная нейронная сеть ResNet~\cite{he2015deep}.
Она позволяет обучать глубокие нейронные сети (до 152 слоев) с высокой точностью,
преодолевая проблему затухания градиента, которая возникает при обучении глубоких сетей.

В последнее время активно ведутся научные исследования,
посвященные методам анализа активности головного мозга и декодирования информации \cite{siuly2016eeg, yuan2021bci, du2022fmri}. Основу применения этих методов составляют технологии нейрокомпьютерных интерфейсов.
Интерфейсы мозг-компьютер или ИМК являются относительно новой, но быстро развивающейся областью исследований методов нейровизуализации \cite{bularka2016brain}.
Технология ИМК открывает новый способ взаимодействия мозга с компьютером.
Система собирает сигналы мозга, анализирует их и преобразует сигналы в команды, которые могут быть отправлены на устройство
вывода для выполнения определенного действия.
Это позволяет осуществлять прямую связь между мозгом и компьютером.
Ключевой целью исследований ИМК является разработка нового канала связи,
который позволяет людям с тяжелыми нервно-мышечными нарушениями напрямую передавать сообщения из своего мозга путем анализа умственной активности.
Классификация моторных образов в задаче ИМК может быть использована, например, для управления протезами или другими устройствами с
помощью мысленных команд \cite{song2020assistive, cruz2021self, schwarz2020decoding}. Такой подход позволяет людям с ограниченными возможностями компенсировать потерю или отсутствие нормальной моторной функции.

Активные области мозга, ответственные за конкретную задачу, называются регионами интереса (англ. ROI~--- Region of Interest). Выделение регионов интереса является важным шагом в предварительной обработке данных ЭЭГ, фМРТ и других методов нейровизуализации, используемых в нейрокомпьютерных интерфейсах \cite{poldrack2007region}. Данные фМРТ представляют собой объемные многомерные наборы данных.
При этом значительная доля вокселей фМРТ соответствует фоновому изображению, которое может вносить существенный шум при решении задачи декодирования сигнала.
Кроме того, необходимо учитывать, что конкретные области мозга отвечают за выполнение определенных зрительных, когнитивных и моторных задач \cite{altenmuller2002brain}.
Таким образом, при анализе снимков фМРТ важным аспектом является корректное выделение регионов интереса, ответственных за выполнение конкретной задачи.

Согласно исследованию \cite{anderson2006}, при фМРТ-обследовании пациентов,
просматривающих видеоряд, активируется определенная корковая сеть. Она включает
в себя, в том числе задние центральные и фронтальные области. Находится эта сеть
преимущественно в правом полушарии.

В настоящей работе рассматривается подход,
использующий для анализа времени задержки именно эти части головного мозга. Также рассматривается задача восстановления зависимости между снимками фМРТ и видеорядом.
Используется предположение, что такая зависимость существует.
Кроме того, предполагается, что между снимком и видеорядом есть постоянная задержка во времени
\cite{logothetis2003underpinnings}.
Проверяется зависимость снимка фМРТ от одного изображения и предыдущего снимка.
Время задержки выступает в качестве гиперпараметра модели.
На основе анализа зависимости предлагается метод аппроксимации показаний фМРТ по
просматриваемому видеоряду.

Кроме того, в настоящей работе в качестве задачи декодирования временных рядов рассматривается задача классификации. Внимание сосредоточено на анализе данных фМРТ для задачи классификации отрезков временного ряда, что может быть полезным для диагностики некоторых неврологических заболеваний \cite{kloppel2012diagnostic}. Изучение активности мозга при обработке визуальной информации позволяет глубже понять механизмы работы мозга и процессы, происходящие в нем. Это, в свою очередь, способствует развитию нейронаук и когнитивной психологии. В работе предлагается метод взвешивания вокселей фМРТ, учитывающий пространственно-временные зависимости в данных. На основе предложенного метода строится модель классификации отрезков временного ряда фМРТ. Модель учитывается пространственные характеристики за счет трансформация пространства признаков в терминах римановой геометрии \cite{barachant2010riemannian, barachant2011multiclass}.
\newpage

