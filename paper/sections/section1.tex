\section{Постановка задачи}\label{sec1}
\subsection{Восстановление снимков фМРТ по видеоряду}
Задана частота кадров $\nu \in \mathbb{R}$ и продолжительность $t \in \mathbb{R}$ видеоряда.
Задан видеоряд
\begin{equation*}
	\label{eq1}
	\bP = [\bp_1, \ldots, \bp_{\nu t}], \quad\
	\bp_{\ell} \in \mathbb{R}^{W \times H \times C},
\end{equation*}
с шириной, высотой и числом каналов изображения $W, H$ и
$C$ соответственно.

Обозначим частоту снимков фМРТ $\mu \in \mathbb{R}$. Задана последовательность снимков
\begin{equation*}
	\label{eq2}
	\bX = [\bx_1, \ldots, \bx_{\mu t}], \quad\
	\bx_{\ell} \in \mathbb{R}^{X \times Y \times Z},
\end{equation*}
где $X, Y$ и $Z$~--- размерности воксельного изображения.

Задача состоит в построении отображения, которое бы учитывало задержку $\Delta t$ между
снимком фМРТ и видеорядом, а также предыдущие томографические показания. Формально, необходимо
найти такое отображение $\mathbf{g}$, что
\begin{equation*}
	\label{eq3}
	\mathbf{g}(\bp_1, \ldots, \bp_{k_{\ell} - \nu \Delta t}; \bx_1, \ldots, \bx_{\ell-1}) = \bx_{\ell},
	\ \ell = 1, \ldots, \mu t,
\end{equation*}
где для $\ell$-го снимка фМРТ номер соответствующего изображения $k_{\ell}$ определяется по формуле
\begin{equation*}
	\label{eq4}
	k_{\ell} = \dfrac{\ell \cdot \nu}{\mu}.
\end{equation*}

\subsection{Классификация временных рядов фМРТ}
\subsubsection{Сегментация временного ряда фМРТ}
\begin{definition}
Фрагментом временного ряда $\bX = [\bx_1, \dots, \bx_T]$ будем называть любую его подпоследовательность $\mathbf{\hat{X}}:$ $$\mathbf{\hat{X}} = [\bx_{(t_1)}, \dots, \bx_{(t_k)}], \quad 1 \leqslant t_1 < \ldots < t_k \leqslant T.$$
\end{definition}
\begin{definition}
Сегментом временного ряда $\bX = [\bx_1, \dots, \bx_T]$ будем называть его непрерывный фрагмент $\mathbf{\hat{X}}:$
$$\mathbf{\hat{X}} = [\bx_{(t)}]_{t=t_0}^{t_1}, \quad \quad 1 \leqslant t_0 < t_1 \leqslant T.$$
Тогда длиной сегмента называется число $\tau = t_1 - t_0 + 1$.
\end{definition}

\begin{definition}
Под сегментацией временного ряда будем понимать отображение $\mathcal{G}$, сопоставляющее временному ряду $\bX$ множество его сегментов $\mathcal{G}(\bX) \in 2^{\mathcal{S}(\bX)}$, где $\mathcal{S}(\bX)$ множество всех сегментов временного ряда $\bX$.
\end{definition}

При анализе медицинских данных важно учитывать уникальные особенности анатомии и реакции каждого человека на стимулы. Для более точного декодирования данных необходимо обучать модель индивидуально под конкретного индивида. В медицинских наборах данных обычно представлено только одно большое измерение для каждого человека. Для решения конкретных задач необходимо разбить или сегментировать временной ряд для дальнейшего анализа. 
\begin{definition}
    Под стимулами будем понимать зрительные, моторные или когнитивные задания различных категорий, которые выполняет человек в ходе процедуры фМРТ.
\end{definition}
Формализуем задачу сегментации. Имеется одно непрерывное измерение фМРТ с дискретным представлением времени:
\begin{equation*}
	\bX = [\bx_{1}, \ldots, \bx_{T}], \quad \bx_{t} \in \mathbb{R}^{X \times Y \times Z},
\end{equation*}
где $X, Y$ и $Z$~--- размерности тензора снимка, $T$~--- длина временного ряда. 
Имеется дискретный временной ряд стимулов:
\begin{equation*}
	\bm{s} = [{s}_{1}, \ldots, {s}_{T}], \quad {s}_{t} \in \{0,\dots, C\},
\end{equation*}
где $\{1,\dots, C\}$~--- множество классов стимулов.
Классом $0$ обозначены моменты отдыха человека.

Необходимо построить сегментацию $\mathcal{G}(\bX, \bm{s})$ временного ряда $\bX$ на сегменты фиксированной длины $\tau$, чтобы в сегментах преобладала определенная категория.

\subsubsection{Классификация сегментов временного ряда фМРТ}
Предположим, что нам дано $N$ сегментов временного ряда фМРТ, каждый из которых имеет длину $T$. Обозначим эти сегменты как:
$$\bX = \{\bX_1, \bX_2, \ldots, \bX_N \},$$
где каждое наблюдение $\bm{X}_i$ представляет собой последовательность воксельных изображений:
$$\bX_i = [\bx_{1}^i, \bx_{2}^i, \ldots, \bx_{T}^i],$$
где $\bx_{t}^i$~--- это воксельное изображение в момент времени $t$ для набора $i$.

Кроме того, каждому наблюдению соответствует метка класса $y_i$, которая может принимать значения от $1$ до $C$, где $C$~--- общее число классов. Обозначаем вектор меток классов как:
$$\bm{y} = [y_1, y_2, \ldots, y_N]^T.$$
Требуется построить модель классификации, которая учитывает\\ пространственно-временные характеристики временных рядов фМРТ. Формально, имеется выборка 
\[ \mathfrak{D} = \{(y_i, \bX_i) \ | \ i = 1, \ldots, N \}. \]
Требуется построить такое отображение $g$, что:
\[g: \bX \rightarrow \{1,\dots, C\}.\] 

\newpage

